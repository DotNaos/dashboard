\section{Discussion}\label{sec:Discussion}
Currently, the application must be opened manually on a phone, tablet, or PC. The next logical step would be to connect a touchscreen to the Raspberry Pi, enabling it to function as a standalone device. This would provide easier, continuous access to the dashboard without relying on external devices.

Another area of improvement is usability and user experience. Features such as GPS-based task autocompletion and integration with external applications like calendars or popular to-do list apps (e.g., Todoist) could make the dashboard a smarter, more centralized tool for users. Enhancing the user interface is also a priority, as the current interface is minimal and lacks features like editing, filtering, and sorting. Adding the option for users to take a photo of food items, with the application using image recognition to identify and automatically add them to the list, would significantly streamline inventory management. Additionally, implementing push notifications for expiring food items would further increase the dashboard's utility.
Beyond enhancing existing functionality, an entirely new feature I considered is an LLM-powered news feed that provides lecture summaries. This feature would run as a scheduled task on the server each morning, scraping course content from Moodle, summarizing it using a large language model (LLM), and displaying the summaries within the application. While a proof of concept has already been implemented, the feature was not included in the final project due to limitations in handling image-based and non-textual content within many current open-source LLM models. Therefore, this idea remains on hold until more advanced LLMs are accessible.
